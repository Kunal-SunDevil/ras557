\documentclass[12pt]{article}
\usepackage{graphicx}
\usepackage{amsmath}
\usepackage{hyperref}
\usepackage{geometry}
\usepackage{enumitem}
\usepackage{float}

\geometry{a4paper, margin=1in}

\title{Foldable Robotics Group Assignment 1}
\author{Group 2: Jahnav Rokalaboina, Kunal Palasdeokar and Nihar Masurkar}
% \date{}

\begin{document}

\maketitle

\section*{1. Discuss and decide on the goals of your project as a team}
Write down your team’s goal.
\begin{enumerate}
    \item Identify a candidate organism you wish to focus on. This consists of a specific animal species, its body plan, and the motion of interest. Search in Google Scholar, using keywords such as “anatomy”, “morphology”, “mechanics”, “biomechanics”, “ground reaction forces”, etc., along with the animal’s informal or scientific name along with the type of locomotion.
    
    \item What would you like to do with this animal? Study some particular aspect? Answer a particular research question? Design a product that addresses a particular need observed?
    
    \begin{center}
        \fbox{\begin{minipage}{0.9\textwidth}
        Please check with Dr. Aukes if you need further clarification / approval.
        \end{minipage}}
    \end{center}
    
    \item Explain your rationale for why it is a good question to ask by answering the following questions. For each numbered answer provide \textit{at least one paragraph per response.}
    
    \begin{enumerate}
        \item \textbf{Scope:} Discuss your plan for constraining the problem to ensure your project can be designed and built within the scope of one semester.
        
        \begin{center}
            \fbox{\begin{minipage}{0.9\textwidth}
            Examples include looking at one specific type of motion, using a specific animal as a source of inspiration (for generating unique specifications), or limiting yourself to a particular set of materials, parts, cost requirements, etc.
            \end{minipage}}
        \end{center}
        
        \item \textbf{Impact:} Is your project interesting, timely, relevant, or important?
        
        \begin{center}
            \fbox{\begin{minipage}{0.9\textwidth}
            Some ways you could answer this question would be by answering one or more of the following questions:
            \begin{itemize}
                \item What makes it important to others?
                \item Why is this idea important now? What prevented it from being answered 10 years ago?
                \item Within what contexts could other people use your results?
                \item What are the potential broader impacts on society?
            \end{itemize}
            \end{minipage}}
        \end{center}
        
        \item \textbf{Team Fit:} How does answering this question leverage your interests and abilities?
        
        \item \textbf{Topic Fit:} How does the question use foldable robotics techniques to answer it?
    \end{enumerate}
\end{enumerate}



\section*{2. Perform Background Research}
\begin{enumerate}
    \item Search for existing research papers (using Google Scholar or similar) on the same animal, subsystem, and motion, as well as robots inspired by this animal or gait.
    
    \item Identify \textbf{four} citations about your selected animal and/or bio-inspired robots based on the same animal, subsystem, and/or motion that will be most helpful in establishing critical design criteria.
    
    \item Collect key information about your organism or its robotic counterpart.
    
    \begin{center}
        \fbox{\begin{minipage}{0.9\textwidth}
        \textbf{Examples of the types of information to include:}
        \begin{itemize}
            \item Typical mass of the animal as a whole, or as the sum of key anatomical parts.
            \item Average speed of the animal.
            \item Key points in a stride: leg motion such as stride length and maximum foot height, trunk motion and orientation, etc.
            \item Typical ground reaction forces during locomotion (a plot is best, so include that as well below the table).
            \item Metabolic energy/power consumed to locomote (respiration).
            \item Mechanical energy/power generated during locomotion.
            \item Key biological materials and their mechanical properties (bone, ligaments, tendons, and the resulting link/joint stiffnesses and damping properties).
            \item Muscle forces.
            \item ...
        \end{itemize}
        \end{minipage}}
    \end{center}
    
    \item Include at least \textbf{two figures} from the literature review you conducted that highlight key aspects of the biological system. This should include one from each of the following categories:
    \begin{enumerate}
        \item Figures/drawings of skeleton, anatomy, exoskeleton, body plan, musculature, kinematics.
        \item Motion plots, freeze frames of gait cycle, plot of ground reaction forces.
        \item ... other aspects of the parameters above.
    \end{enumerate}
    
    \item Discuss these papers, highlighting the information you can draw from each. Be specific. Why is each paper valuable? (At least one paragraph per paper)
    
    \item From what you have found, how your project topic novel? (one paragraph)
\end{enumerate}



\section*{3. Estimate Goal Performance Metrics}
Identify physical metrics of your design that are missing from background research, or from separate sources. Use your existing knowledge of statics, dynamics, materials, and physics to address these missing pieces.

\begin{center}
    \fbox{\begin{minipage}{0.9\textwidth}
    \textbf{Example 1:} If you know the vertical ground reaction forces during a jump, and the animal’s mass, use $F = m/a$ to establish the acceleration of the animal.
    \end{minipage}}
\end{center}

\begin{center}
    \fbox{\begin{minipage}{0.9\textwidth}
    \textbf{Example 2:} If you know the maximum height an animal can jump along with its mass and some time-based information about its jump, you could use
    \[
    mgh = \frac{1}{2}mv_{1}^2
    \]
    to find the velocity of the animal as it leaves the ground and
    \[
    F = ma \tag{1}
    \]
    \[
    a = F/m \tag{2}
    \]
    \[
    v(t) = (F/m)t + v_0 \tag{3}
    \]
    \[
    v_1 = (F/m)t + 0 \tag{4}
    \]
    to find the average vertical ground reaction force over time $t$ required to reach that height.
    \end{minipage}}
\end{center}




\section*{4. Assemble a Specifications Table}
Collect all the physics-based information you have found from your references into one place. A well-formatted table may do, with supplementary figures from literature as needed. A specifications table is a handy way to collect parameters. Use SI units. Example below:

\begin{table}[h]
\centering
\begin{tabular}{|l|c|c|c|}
\hline
\textbf{Parameter} & \textbf{Unit} & \textbf{Value Range} & \textbf{Reference} \\
\hline
Total Mass & kg & 0.2 - 4 & [1] \\
Maximum Landing Force & N & 6 & calculated above \\
Maximum Takeoff Force & N & 4 & [2] \\
Average Takeoff acceleration & $\frac{m}{s^2}$ & 13 & [3] \\
... & ... & ... & ... \\
\hline
\end{tabular}
\caption{Example of Specifications Table}
\end{table}

\begin{enumerate}
    \item Make sure what source you got each piece of information from, or whether you calculated it yourself.
    \item Discuss any changes you made to the values found in your background reading, for safety or performance reasons.
\end{enumerate}




\section*{5. Develop a Kinematic model}
Develop a Kinematic model for your device using a Python-based script. You may use a vector-based approach, a quaternion-based approach, or a MuJoCo-based approach.

\begin{center}
    \fbox{\begin{minipage}{0.9\textwidth}
    Kinematics implies that you will not be considering system stiffness, mass, or motion as a function of time yet. Thus, if you use MuJoCo, use it for its kinematics abilities rather than for the full system.
    \end{minipage}}
\end{center}

\begin{enumerate}
    \item Define all variables and constants (especially if they were not defined in your figure).
    \item Declare frames and rotation tables.
    \item Create vector descriptions for key points.
    \item Identify a set of important mechanism states (or configurations) that represent the system at important parts of a typical gait, when it is both moving and when forces are being applied to it (or to the world by it).
    \item Plot the system in these states.
    \item Identify the system Jacobian symbolically or numerically, and use it at key points to:
    \begin{enumerate}
        \item \textbf{Understand the force relationships between outputs and inputs.} From your biomechanics-based specifications, define one or more estimates for ground reaction forces that the system should be expected to experience.
        \begin{itemize}
            \item Consider including, based on your background research:
            \begin{itemize}
                \item the force of gravity exerted by the mass of a “payload” or the main body of the robot.
                \item the acceleration the system experiences during a typical gait.
                \item ground reaction forces measured from biomechanics studies.
            \end{itemize}
            \item Calculate the force or torque required at the actuators to satisfy the end-effector force requirements.
        \end{itemize}
        
        \item \textbf{Understand the velocity relationships between inputs and outputs.} Estimate the velocity of the end-effector in key configurations. Using the Jacobian, calculate the speed required by the input(s) to achieve that output motion.
    \end{enumerate}
    
    \begin{center}
        \fbox{\begin{minipage}{0.9\textwidth}
        This may not be directly solvable based on your device kinematics; an iterative guess-and-check approach is ok.
        \end{minipage}}
    \end{center}
    
    \item Finally, using the two calculations for force and speed at the input, compute the required power in this specific state.

    \item Discussion:
    \begin{enumerate}
        \item How many degrees of freedom does your device have? How many motors? If the answer is not the same, what determines the state of the remaining degrees of freedom? How did you arrive at that number?
        \item How did you estimate your expected end-effector forces?
        \item How did you estimate your expected end-effector speeds?
    \end{enumerate}
\end{enumerate}


\section*{6. Develop an analogous mechanism}
Develop an analogous mechanism for your source of bio-inspiration in paper or cardboard.
\begin{enumerate}
    \item \textbf{Make the mechanism.} It can be cut by hand, but must clearly communicate where motor(s) will be connected, what touches the ground, and be able to move through its proposed range of motion from (only) motion about the actuators. Take pictures of it in multiple configurations of its gait cycle.
    \begin{itemize}
        \item The mechanism should use a link / joint topology and the folding approach for making joints that you have learned from this class.
        \item Consider using parallel and series mechanisms as they are best suited.
        \item Consider both planar and spherical mechanisms as appropriate for your needs.
        \item Consider where to eventually place springs.
    \end{itemize}
    
    \item \textbf{Draw the proposed mechanism.} Use a vector drawing program (Inkscape, IPE, Illustrator, Draw.io) or 3D modeling software to ensure a professional result.
    \begin{itemize}
        \item Label each reference frame, including your inertial (Newtonian) reference frame.
        \item Include a set of orthonormal basis vectors for each frame ($\hat{n}_x, \hat{n}_y, \hat{n}_z$), ($\hat{a}_x, \hat{a}_y, \hat{a}_z$). It is best practice to align one of the basis vectors (like $\hat{a}_x$) with each rigid link.
        \item Variable names for each state variable ($\theta_1, \theta_2, \dot{\theta}_1, \dot{\theta}_2, ...$)
        \item Geometric values such as link lengths ($l_1, l_2, ...$)
    \end{itemize}
    
    \begin{center}
        \fbox{\begin{minipage}{0.9\textwidth}
        \textbf{Save this figure for reuse later.} You will need to add mass and inertial information as well as system stiffness information, so make sure you do your work in a way that permits reusing and modifying the figure.
        \end{minipage}}
    \end{center}
\end{enumerate}



\section*{7. System Identification}
The purpose of this part is to create a plan for identifying key model parameters that teammates will be able to execute individually, and then contribute back to their team’s modeling effort. This should include aspects from the following:
\begin{itemize}
    \item Actuator parameters – resistance, inductance, inertia, damping, mass, $K_v / K_\tau$, gearing – fit against experimental data.
    \item The mass and inertia properties of key parts of your proposed system, modeled and verified.
    \item The stiffness and damping characteristics:
    \begin{itemize}
        \item of your team’s joints, as fabricated.
        \item of your team’s links, as fabricated.
        \item of a key subsystem, as fabricated.
        \item of a discrete energy storage component.
    \end{itemize}
    \item Friction estimations and a model fitting process between key materials undergoing contact interactions.
    \item ...other ideas approved by your professor.
\end{itemize}

\begin{enumerate}
    \item Identify and discuss the various parameters you plan to model in your simulation. Discuss your plans for experimentally obtaining each of those values, and the model you would like to use for describing each phenomenon.
    
    \item Create a table with at least four system identification experiments you will run, and an assigned team lead for each experiment.
\end{enumerate}

\begin{table}[h]
\centering
\begin{tabular}{|l|c|c|c|c|}
\hline
\textbf{Item} & \textbf{Sam} & \textbf{Jo} & \textbf{Pat} & \textbf{Stevie} \\
\hline
Servo Characterization & x &  &  &  \\
Link Stiffness Experiment &  & x &  &  \\
Joint Stiffness Experiment &  &  & x &  \\
Mass and inertia calculations and validation of main body &  &  &  & x \\
\hline
\end{tabular}
\caption{System Identification Experiments and Assigned Team Leads}
\end{table}

\noindent Individual experiments will be assigned to each team member as part of an upcoming individual assignment.


\section*{8. Project Part 2 Roadmap}
Finally, plan the upcoming tasks and roles for the other team activities. Write up your plan for the following:
\begin{enumerate}
    \item Identify and discuss the materials you plan to use in fabrication, and key design decisions, such as how you plan to fabricate parts. Decide who will be obtaining those materials and distributing them.
    
    \item Identify and discuss how you plan to prototype your system and assign one person to do that.
    
    \item Identify and discuss how you will collect system-level motion or force data for validation, including:
    \begin{itemize}
        \item method (IMU, video, discrete joint sensors, force/torque sensing)
        \item data extraction approach
        \item how you plan to characterize performance. What metrics will you use and what experiments do you need to measure that performance?
        \item how you plan to visualize your data.
    \end{itemize}
    
    \begin{center}
        \fbox{\begin{minipage}{0.9\textwidth}
        \textbf{Note:} We are distinguishing from component level model fitting, which should come before you add elements to your model, and verification / validation experiments, which should come after you have built your final device.
        \end{minipage}}
    \end{center}
    
    \item Identify and discuss your plan for shared simulation tasks:
    \begin{itemize}
        \item who will be working with the code
        \item adding model fitting routines
        \item filtering, interpolating, and otherwise massaging input data
        \item optimization approach
    \end{itemize}
    
    \item Identify and discuss any reporting tasks that may be needed:
    \begin{itemize}
        \item compiling information into a report (may be combined with the simulation if using Jupyter)
        \item managing the GIT repository
    \end{itemize}
    
    \item Split each of these tasks to the individuals on your team and visualize task assignments in a table.
\end{enumerate}



\begin{itemize}
    \item Project Definition
    \item Background Research
    \item Initial Calculations
    \item Specifications Table
    \item Modeling and Analysis
    \item Mechanism Prototype and Figure
    \item System Identification Plan and Roadmap
\end{itemize}


\bibliographystyle{IEEEtran}
\bibliography{IEEEabrv,reference}

\end{document}